\documentclass[a4paper,12pt,titlepage,finall]{article}

\usepackage[T1,T2A]{fontenc}     % форматы шрифтов
\usepackage[utf8x]{inputenc}     % кодировка символов, используемая в данном файле
\usepackage[russian]{babel}      % пакет русификации
\usepackage{tikz}                % для создания иллюстраций
\usepackage{pgfplots}            % для вывода графиков функций
\usepackage{geometry}		 % для настройки размера полей
\usepackage{indentfirst}         % для отступа в первом абзаце секции
\usepackage{multirow}            % для таблицы с результатами
\usepackage{hyperref}            % для нажимаемых ссылок

% выбираем размер листа А4, все поля ставим по 3см
\geometry{a4paper,left=30mm,top=30mm,bottom=30mm,right=30mm}

\setcounter{secnumdepth}{0}      % отключаем нумерацию секций

\usepgfplotslibrary{fillbetween} % для изображения областей на графиках

\begin{document}
% Титульный лист
\begin{titlepage}
    \begin{center}
	{\small \sc Московский государственный университет \\имени М.~В.~Ломоносова\\
	Факультет вычислительной математики и кибернетики\\}
	\vfill
	{\Large \sc Отчет по заданию №1}\\
	~\\
	{\large \bf <<Методы сортировки>>}\\ 
	~\\
	{\large \bf Вариант 1 / 1 / 2 / 4}
    \end{center}
    \begin{flushright}
	\vfill {Выполнила:\\
	студентка 104 группы\\
	Соколова~А.~О.\\
	~\\
	Преподаватель:\\
	Гуляев~Д.~А.}
    \end{flushright}
    \begin{center}
	\vfill
	{\small Москва\\2022}
    \end{center}
\end{titlepage}

% Автоматически генерируем оглавление на отдельной странице
\tableofcontents
\newpage

\section{Постановка задачи}

Используя язык программирования C реализовать два метода сортировки: метод простого выбора и быструю сортировку. Провести экспериментальное сравнение их эффективности в решении задачи сортировки по неубыванию. Сравнение должно быть проведено на разных массивах целых чисел (int). При выполнении сортировок на каждом массиве подсчитать число сравнений и число перемещений элементов.

\newpage

\section{Результаты экспериментов}

Сложность выполнения сортировки методом простого выбора в лучшем, среднем и худшем случаях - $O(n^2)$. Количество выполняемых сравнений равно $\frac{n*(n-1)}{2}$, количество выполняемых перемещений равно $n-1$~\cite{bookcs}.

Общее число сравнений быстрой сортировки в лучшем случае равно $n*log_2 n$, а число перемещений $\frac{n*log_2 n}{6}$. При средней производительности число сравнений и перемещений умножается на $2*ln 2$. В худшем случае производительность метода будет порядка $n^2$~\cite{bookcs}.

Результаты эксперимента приведены ниже в таблицах 1 и 2. В моих результатах общее количество операций, выполняемых методом простого выбора, всегда больше общего количества операций, выполняемых быстрой сортировкой на том же массиве. При этом количество операций перемещения у метода простого выбора всегда не больше, чем у быстрой сортировки. Это соответствует приведенным формулам.

\begin{table}[h] 
\centering
\begin{tabular}{|c|c|c|c|c|c|c|c|}
    \hline
    \multirow{2}{*}{\textbf{n}} & \multirow{2}{*}{\textbf{Параметр}} & \multicolumn{4}{|c|}{\textbf{Номер сгенерированного массива}} & \textbf{Среднее} \\
    \cline{3-6}
    & & \parbox{1.5cm}{\centering 1} & \parbox{1.5cm}{\centering 2} & \parbox{1.5cm}{\centering 3} & \parbox{1.5cm}{\centering 4} & \textbf{значение} \\
    \hline
    \multirow{2}{*}{10} & Сравнения & 45 & 45 & 45 & 45 & 45\\
    \cline{2-7}
                        & Перемещения & 9 & 9 & 9 & 9 & 9 \\
    \hline
    \multirow{2}{*}{100} & Сравнения & 4950 & 4950 & 4950 & 4950 & 4950 \\
    \cline{2-7}
                         & Перемещения & 99 & 99 & 99 & 99 & 99 \\
    \hline
    \multirow{2}{*}{1000} & Сравнения & 499500 & 499500 & 499500 & 499500 & 499500 \\
    \cline{2-7}
                          & Перемещения & 999 & 999 & 999 & 999 & 999 \\
    \hline
    \multirow{2}{*}{10000} & Сравнения & 49995000 & 49995000 & 49995000 & 49995000 & 49995000 \\
    \cline{2-7}
                           & Перемещения & 9999 & 9999 & 9999 & 9999 & 9999 \\
    \hline
\end{tabular}
\caption{Результаты работы сортировки методом простого выбора}
\end{table}

\begin{table}[h]
\centering
\begin{tabular}{|c|c|c|c|c|c|c|c|}
    \hline
    \multirow{2}{*}{\textbf{n}} & \multirow{2}{*}{\textbf{Параметр}} & \multicolumn{4}{|c|}{\textbf{Номер сгенерированного массива}} & \textbf{Среднее} \\
    \cline{3-6}
    & & \parbox{1.5cm}{\centering 1} & \parbox{1.5cm}{\centering 2} & \parbox{1.5cm}{\centering 3} & \parbox{1.5cm}{\centering 4} & \textbf{значение} \\
    \hline
    \multirow{2}{*}{10} & Сравнения & 19 & 12 & 15 & 12 & 14\\
    \cline{2-7}
                        & Перемещения & 6 & 11 & 10 & 11 & 9 \\
    \hline
    \multirow{2}{*}{100} & Сравнения & 480 & 386 & 441 & 429 & 434 \\
    \cline{2-7}
                         & Перемещения & 63 & 112 & 182 & 180 & 134 \\
    \hline
    \multirow{2}{*}{1000} & Сравнения & 7987 & 6996 & 7557 & 7809 & 7587 \\
    \cline{2-7}
                          & Перемещения & 511 & 1010 & 2597 & 2631 & 1687 \\
    \hline
    \multirow{2}{*}{10000} & Сравнения & 113631 & 103644 & 105709 & 96732 & 104929 \\
    \cline{2-7}
                           & Перемещения & 5904 & 10904 & 33585 & 33949 & 21085 \\
    \hline
\end{tabular}
\caption{Результаты работы быстрой сортировки}
\end{table}

\newpage

\section{Структура программы и спецификация функций}

Функции сортировки:
\begin{enumerate} 
  \item \emph{simple\_choice(int n, int *arr, int *count\_comp, int *count\_move)} – функция, которая сортирует массив методом простого выбора и считает число сравнений и перемещений элементов. Функция не имеет возвращаемых значений и принимает в качестве параметров размер массива, указатель на массив, указатели на переменные, которые отвечают за число сравнений и перемещений.  
  \item \emph{quick\_sort(int n, int *arr, int *count\_comp, int *count\_move)} – функция, которая сортирует массив методом быстрой сортировки и считает число сравнений и перемещений элементов. Функция не имеет возвращаемых значений и принимает в качестве параметров размер массива, указатель на массив, указатели на переменные, которые отвечают за число сравнений и перемещений.
\end{enumerate}

Функции для генерации массивов:
\begin{enumerate}
    \item \emph{ascending(int n, int *arr)} – функция, которая заполняет массив размера n числами от 1 до n в порядке возрастания. Функция не имеет возвращаемых значений и принимает в качестве параметров размер массива, указатель на массив.
    \item \emph{descending(int n, int *arr)} – функция, которая заполняет массив размера n числами от n до 1 в порядке убывания. Функция не имеет возвращаемых значений и принимает в качестве параметров размер массива, указатель на массив.
    \item \emph{random\_array(int n, int *arr)} – функция, которая заполняет массив размера n рандомными числами, которые генерирует с помощью функциии \emph{rand}. Функция не имеет возвращаемых значений и принимает в качестве параметров размер массива, указатель на массив.
     \item \emph{generator(int n, int *arr, int par)} – функция, генерирующая массив определенного типа, который определяется в соответствии с переданным параметром. Функция не имеет возвращаемых значений и принимает в качестве параметров размер массива, указатель на массив и параметр, отвечающий за тип массива.
\end{enumerate}

\newpage

\section{Отладка программы, тестирование функций}

Функции сортировки были реализованы в соотвествии с описанием в учебных материалах~\cite{bookcs}, их работа была проверена на трёх самостоятельно выбранных массивах. Другие функции так же были протестированы на трёх различных наборах входных параметров.

\section{Анализ допущенных ошибок}

В ходе работы над проектом было допущено три существенные ошибки:
\begin{enumerate}
\item Ключ для рандомной генерации передавался в функцию \emph{srand} внутри функции \emph{generator}, из-за этого все рандомно сгенерированные массивы содержали одинаковую последовательность элементов. Данная ошибка была исправлена путём перемещения вызова функции \emph{srand} в тело \emph{main}.
\item Сгенерированный массив не копировался для выполнения второй сортировки на нём, из-за этого быстрая сортировка выполнялась на уже отсортированном массиве. Данная ошибка была исправлена с помощью копирования созданного массива и использования второго указателя при выполнении второго метода сортировки.
\item Перемещение элементов было неправильно расположено в коде функции \emph{simple\_choice}. Код \emph{simple\_choice} был исправлен в соответствии с алгоритмом, описанным в методических материалах. 
\end{enumerate}

Все ошибки были выявлены на этапе получения результатов и сверке их с формулами из учебных материалов.


\newpage
\begin{thebibliography}{1}
    \bibitem{bookcs}
        Вирт Н. Алгоритмы и структуры данных. — М.: Мир, 1989
\end{thebibliography}
\end{document}
